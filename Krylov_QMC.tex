% Document template for ANS Journals
% Options: footnoteAtEnd - Places all footnotes at the end of document
%               Usage: \documentclass[footnoteAtEnd]{style/nseJournal}
\documentclass{nseJournal}
\usepackage{local}

\begin{document}

\title{Krylov Linear Solvers and Quasi Monte Carlo
Methods for Transport Simulations}

% Use the \addAuthor macro to add authors in the order they should appear. The second argument corresponds to
% the affiliation declared below.
\addAuthor{Sam Pasmann}{a}
\addAuthor{C. T. Kelley}{b}
\addAuthor{\correspondingAuthor{Ryan McClarren}}{a}
\correspondingEmail{rmcclarr@nd.edu}

% Affiliations can be added in the order they should appear. For breaks in addresses, use either \\ or \tabularnewline
\addAffiliation{a}{Department of Aerospace and Mechanical Engineering\\ University of Notre Dame\\ Fitzpatrick Hall, Notre Dame, IN 46556}
\addAffiliation{b}{North Carolina State University, Department of
Mathematics\\ 3234 SAS Hall, Box 8205\\ Raleigh NC 27695-8205}

% Add keywords to appear in Abstract in the order they should appear
\addKeyword{Quasi Monte Carlo Methods}
\addKeyword{Krylov Linear Solvers}

\titlePage

\begin{abstract}
The Boltzmann form of the neutron transport equation describes the behavior of neutrons as they collide from one atom to another and is important in nuclear reactor design and accident analysis. The full neutron transport equation is space, direction, time and energy dependent. In many cases an analytic solution cannot be solved unless many simplifying assumptions are made. Numerical methods must be relied upon to provide physically realistic solutions. Iterative methods, like the diffusion accelerate Source Iteration (SI) are a standard practice. These iterative methods can be enhanced by replacing the deterministic transport sweep with Monte Carlo (MC) simulation. This Hybrid method can be further enhanced using Quasi-Monte Carlo (QMC) techniques, reducing the noise inherent in MC simulations. However, the SI method can suffer from low convergence rates, particularly in optically thick media. Preconditioned Krylov Subspace methods provide a more efficient solution compared to the SI method. Krylov methods including GMRES and BiCGStab are shown to outperform the SI in several test problems using the outlined hybrid QMC-iterative solver method.  
\end{abstract}


\section{Introduction}
\label{sec:intro}


\section{Computational Results}
\label{sec:results}
In this section we consider an example from
\cite{cesinh}. The formulation of the transport
problem is taken from \cite{ctk:jeff1}. The equation for the angular
flux \(\psi\) is

\[
\mu \frac{\partial \psi}{\partial x} (x,\mu) + \Sigma_t(x) \psi(x,\mu) =
\frac{1}{2} \left[ \Sigma_s(x) \int_{-1}^1 \psi(x, \mu') \dmup + q(x) \right]
 \mbox{ for } 0 \le x \le \tau
\]

The boundary conditions are

\[
\psi(0, \mu) = \psi_l(\mu), \mu > 0; \psi(\tau, \mu) = \psi_r(\mu),
\mu < 0.
\]

The notation is

\begin{itemize}
\item
  \(\psi\) is intensity of radiation or angular flux at point \(x\) at
  angle \(\cos^{-1} (\mu)\)
\item
  \(\phi = \phi(x) = \int_{-1}^{1}\psi(x,\mu) \ d\mu\) is the scalar
  flux, the \(0^{th}\) angular moment of the angular flux. -
  \(\tau < \infty\), length of the spatial domain. -
  \(\Sigma_s \in C([0,\tau])\) is the scattering cross section at \(x\)
  - \(\Sigma_t \in C([0,\tau])\) is the total cross section at \(x\) -
  \(\psi_l\) and \(\psi_r\) are incoming intensities at the bounds -
  \(q \in C([0,\tau])\) is the fixed source
\end{itemize}


\subsection{The Garcia-Siewert Example}
\label{the-garcia-siewert-example}

In this example \[
\tau=5, \Sigma_s(x) =\omega_0 e^{-x/s},  \Sigma_t(x) = 1, q(x) = 0, \psi_l(\mu) = 1, \psi_r(\mu) = 0.
\]

\subsection{Solvers}
\label{subsec:solvers}

The linear and nonlinear solvers come from the Julia package
\href{https://github.com/ctkelley/SIAMFANLEquations.jl}{SIAMFANLEQ.jl}
\cite{ctk:siamfanl}. The documentation for these codes is in the
\href{https://github.com/ctkelley/NotebookSIAMFANL}{Juila notebooks}
that accompany the package \cite{ctk:notebooknl}. 
These are part of a book project \cite{ctk:fajulia}. 

\subsection{Source Iteration and Krylov Methods}
\label{source-iteration-and-krylov-methods}

We use two krylov methods \cite{ctk:roots}, GMRES \cite{gmres} and
Bi-CGSTAB \cite{bicgstab}.

\subsection{QMC}\label{qmc}

\subsection{Validation and calibration study}
\label{validation-and-calibration-study}

I'll compare the results from the SN computation to what I get from
Sam's QMC code. My SN results are for a very fine spatial mesh and fine
enough angular mesh. They are good to at least six figures and I will
regard them as exact for this study.

I will use the SN results for the table in the Garcia-Siewert paper and
get results from QMC in the following way

\begin{itemize}
\item
  For a give N and Nx I will get cell average fluxes from Sam's code.
\item
  I will use the same code to generate the tables that I used for the SN
  fluxes. That code is \textbf{src/sn\_tabulate.jl}
\item
  I will take the two 11 x 2 arrays of results DataSN and DataQMC and
  compoute componentwise relative error with
\end{itemize}

\begin{verbatim}
Derr = (DataSN-DataQMC)./DataSN
\end{verbatim}

\subsection{QMC and Krylov Linear Solvers}
\label{qmc-and-krylov-linear-solvers}

I can solve the QMC linear problem with both Krylov methods now and,
just like the classical case, I'm seeing fewer than half of the number
of transport sweeps. Herewith the results for N=1000 and Nx= 100.







\clearpage


\section{Conclusion}
\label{sec:conclusion}


\pagebreak
\section*{Acknowledgments}

The research of CTK was supported by 
Department of Energy grant DE-NA003967,
%
and National Science Foundation Grants
% RTG
DMS-1745654,
and
% New NSF
DMS-1906446.


\pagebreak
\bibliographystyle{ans_js}                                                                           %custom ANS journal submission template bibliography style
\bibliography{qmckrylov}

\end{document}


