\section{Algorithm}
\label{sec:algorithm}




 %%%%%%%%%%%%%%%%%%%%%%%%%%%%%%%%%%%%%%%%%%%%%%%%%%%%%%%%%%%%%%%%%%
 %%%%%%%%%%%%%%%%%%%%%%%%%%%%%%%%%%%%%%%%%%%%%%%%%%%%%%%%%%%%%%%%%%
The program was written in Julia, a scientific computing language that combines the compiler capabilities of C++ and the syntax of Matlab and Python. The code and primary documentation are available here \href{https://github.com/ctkelley/Krylov_QMC} {Krylov\textunderscore QMC} \cite{ctk:krylovqmc}. The linear and nonlinear solvers come from the Julia package \href{https://github.com/ctkelley/SIAMFANLEquations.jl}{SIAMFANLEQ.jl} \cite{ctk:siamfanl}. The documentation for these codes is in the \href{https://github.com/ctkelley NotebookSIAMFANL}{Juila notebooks} \cite{ctk:notebooknl} and the book \cite{ctk:fajulia} that accompany the package. 

\noindent\begin{minipage}{\textwidth}
\begin{minipage}{0.45\textwidth}
		\centering
		\captionof{algorithm}{Source Iteration}
		\label{alg:SI}
		\begin{algorithmic}[1]
		\STATE Initialize $\phi_0$
		\WHILE{($r>\textrm{tolerance}$)}
			\STATE $q = \phi_{n-1}*\Sigma_{s} + \textrm{source}$
			\STATE $\phi_n = \mbox{QMC Sweep}(q)$
			\STATE $r_{i} = \frac{||\phi_n - \phi_{n-1}||}{||\phi_{n-1}||}$
		\ENDWHILE
		\STATE Return: $\phi$
	\end{algorithmic} 
\end{minipage}
\hfill
\begin{minipage}{0.45\textwidth}
	\centering
	\captionof{algorithm}{QMC Sweep $(q)$}
	 \label{alg:qmc_sweep}
	\begin{algorithmic}[1]
		\STATE Initialize Low-Discrepancy-Sequence
		\FOR {$i$ in $N$}
			\STATE Assign $r_i$ and $\mu_i$
			\FOR {$j$ in Zones}
				\STATE Move particle across $\textrm{Zone}_j$
				\STATE Tally($r, \mu, weight$)
			\ENDFOR
		\ENDFOR
		\STATE Return: $\phi$
	\end{algorithmic} 
\end{minipage}
\end{minipage}



